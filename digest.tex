\documentclass[conference]{IEEEtran}
\IEEEoverridecommandlockouts

\usepackage{cite}
\usepackage{amsmath,amssymb,amsfonts}
\usepackage{algorithmic}
\usepackage{graphicx}
\usepackage{textcomp}
\usepackage{xcolor}
\usepackage{hyperref}

\def\BibTeX{{\rm B\kern-.05em{\sc i\kern-.025em b}\kern-.08em
    T\kern-.1667em\lower.7ex\hbox{E}\kern-.125emX}}

\begin{document}

\title{A Systematic Methodology for DC Short-Circuit Protection Evaluation in High-Power Data Center Microgrids}

\author{\IEEEauthorblockN{Author Name\IEEEauthorrefmark{1},
ABB Researcher Name\IEEEauthorrefmark{2}}
\IEEEauthorblockA{\IEEEauthorrefmark{1}Organization Name\\
Email: [email protected]}
\IEEEauthorblockA{\IEEEauthorrefmark{2}ABB Corporate Research\\
Email: [email protected]}}

\maketitle

\begin{abstract}

\end{abstract}

\begin{IEEEkeywords}
DC microgrids, short-circuit protection, solid-state circuit breakers, data centers, fault analysis, protection coordination
\end{IEEEkeywords}

\section{Introduction}

\subsection{Motivation}

\subsection{Technical Challenges}

\subsection{Research Gap and Contribution}

\section{Methodology}

\subsection{System Modeling Framework}

\subsubsection{Component-Level Models}

\subsubsection{Fault Scenario Generation}

\subsection{Transient Simulation Approach}

\subsubsection{Multi-Physics Modeling}

\subsubsection{Source Contribution Analysis}

\subsection{Protection Coordination Strategy}

\subsubsection{Selectivity Analysis}

\subsubsection{Device Selection Criteria}

\subsection{Validation Framework}

\section{Case Study: 800V DC Microgrid}

\subsection{System Architecture}

\subsection{Simulation Results}

\subsubsection{Fault Current Analysis}

\subsubsection{Protection Device Evaluation}

\subsection{Laboratory Validation}

\subsection{Key Findings}

\section{Discussion}

\subsection{Implications for Data Center Design}

\subsection{Technology Readiness}

\subsection{Standardization Needs}

\section{Conclusion}

\section*{Acknowledgment}

\bibliographystyle{IEEEtran}
\bibliography{references}

\end{document}
