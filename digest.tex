\documentclass[conference]{IEEEtran}
\IEEEoverridecommandlockouts

\usepackage{cite}
\usepackage{amsmath,amssymb,amsfonts}
\usepackage{algorithmic}
\usepackage{graphicx}
\usepackage{textcomp}
\usepackage{xcolor}
\usepackage{hyperref}

\def\BibTeX{{\rm B\kern-.05em{\sc i\kern-.025em b}\kern-.08em
    T\kern-.1667em\lower.7ex\hbox{E}\kern-.125emX}}

\begin{document}

\title{Scalable Approach to Calculating Short-Circuit Currents for Gigawatt-scale DC Data Centers}

\author{\IEEEauthorblockN{Giel Van den Broeck\IEEEauthorrefmark{1},
Pavel Purgat\IEEEauthorrefmark{2},
Dusan Brhlik\IEEEauthorrefmark{1},
Antonello Antoniazzi\IEEEauthorrefmark{2}}
\IEEEauthorblockA{\IEEEauthorrefmark{1}Direct Energy Partners\\
Email: \{giel.vandenbroeck, dusan.brhlik\}@directenergypartners.com}
\IEEEauthorblockA{\IEEEauthorrefmark{2}ABB\\
Email: pavel.purgat@at.abb.com, antonello.antoniazzi@it.abb.com}}

\maketitle

\begin{abstract}

\end{abstract}

\begin{IEEEkeywords}
DC microgrids, short-circuit protection, solid-state circuit breakers, data centers, fault analysis, protection coordination
\end{IEEEkeywords}

\section{Introduction}

% §1: Industry Trend
% - DC distribution emerging as key enabler for megawatt-scale racks in AI/HPC data centers
% - Power densities pushing beyond 100 kW per rack toward MW-level compute infrastructure
% - Battery storage integrated to buffer rapid power fluctuations — avoiding causing headache for grid operators
% - Architectures at 400V and 800V with multi-source topologies for backup and fluctuations
%
% §2: Where Existing Standards Fall Short
% - IEC 61660 (1997): calculation method for batteries, rectifiers, DC motors, capacitors
% - Gap 1: DC/DC converters have no defined model — despite being central to modern DC systems
% - Gap 2: No methodology to calculate voltage as function of time post-fault — critical for system uptime
% - Standard designed for auxiliary systems, not megawatt-scale DC microgrids
%
% §3: The Gap — What Is Missing
% - No methodology to perform short-circuit current calculations for complex DC systems with numerous converters, batteries, and charged capacitors
% - Cannot prove by calculation that protection devices handle fault current levels
% - Cannot verify protection acts fast enough to prevent voltage collapse
% - Cannot verify selectivity of protection devices
% - Essential for: sizing equipment, specifying protection devices, enabling inspection bodies to verify compliance

% §4: Technical Challenge
% - DC fault behavior fundamentally differs from AC: exponential current rise, no natural zero crossing
% - Each source type (battery, converter, capacitor) contributes with distinct transient characteristics
% - Voltage collapse: limited energy in capacitors causes voltage to collapse within milliseconds — protection must act in sub-ms to single-digit ms range
% - Availability: fast fault clearing essential for system uptime, preventing cascading failures
% - Selectivity: protection scheme must isolate only the faulted section while maintaining power to unaffected loads
% - Safety advantage: fast solid-state interruption significantly reduces arc flash hazards compared to AC systems
%
% §5: Need for Practical Accessibility
% - Calculation method must be usable by installers, system designers, and inspection bodies
% - Applicable in modern software packages without requiring expertise in differential equations
% - Industry hesitates to adopt DC at scale due to safety concerns stemming from absence of established calculation methods and protection guidelines

% §6: Contribution
% - Systematic simulation-based methodology for DC short-circuit current calculations
% - Enables specifying and selecting protection devices based on calculated peak current, rise time and I²t
% - Validated approach for protection device sizing and withstand verification
% - Applicable to complex multi-source DC topologies at megawatt scale
% - Purpose: provide the essential stepping stone to prove protection adequacy by calculation
%
% §7: Paper Outline Preview
% - Section II: Methodology — system modeling, fault scenarios, transient metrics, coordination strategy
% - Section III: Case study — 800V DC datacenter architecture with laboratory validation
% - Section IV: Discussion — design implications, technology readiness, standardization needs
% - Section V: Conclusion — summary and call for standardization collaboration

\section{Methodology}

\subsection{System Modeling Framework}

\subsubsection{Component-Level Models}

\subsubsection{Fault Scenario Generation}

\subsection{Transient Simulation Approach}

\subsubsection{Source Contribution Analysis}

\subsection{Protection Coordination Strategy}

\subsubsection{Selectivity Analysis}

\subsubsection{Device Selection Criteria}

\subsection{Validation Framework}

\section{Case Study: 800V DC Datacenter Architecture}

\subsection{System Architecture}

\subsection{Simulation Results}

\subsubsection{Fault Current Analysis}

\subsubsection{Protection Device Evaluation}

\subsection{Laboratory Validation}

\subsection{Key Findings}

\section{Discussion}

\subsection{Implications for Data Center Design}

\subsection{Technology Readiness}

\subsection{Standardization Needs}

\section{Conclusion}

\section*{Acknowledgment}

\bibliographystyle{IEEEtran}
\bibliography{references}

\end{document}
